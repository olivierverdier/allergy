\documentclass{scrartcl}

\usepackage[%
watermark, % date and time of the draft as watermark
showeqnr, % shows all equation numbers
theoremdefs, % standard theorems and definitions
% final, % activate this option when the document is ready
]{allergy}

\usepackage{listings}
\lstset
{
    language={[LaTeX]TeX},
    basicstyle=\ttfamily,
    columns=flexible,
}


\title{Allergy: a \LaTeX{} metapackage}
\author{Olivier Verdier}

\begin{document}

\maketitle

I always find myself loading the same basic packages over and over again in all my notes and articles.
Over the years, the loaded packages have stabilised into what is now \texttt{allergy.sty}.

It can be used with different options at the three production stages of any document, the \emph{drafting}, \emph{review} and \emph{publication} stages.

\begin{enumerate}
  \item
A typical use case at \textbf{draft stage} is as follows:
Load the \texttt{allergy} package as
\begin{lstlisting}
\usepackage[%
watermark, % date and time of the draft as watermark
showeqnr, % shows all equation numbers
theoremdefs, % standard theorems and definitions
% final, % activate this option when the document is ready
]{allergy}
\end{lstlisting}
\item
At the \textbf{review stage}, you can drop the \lstinline|watermark| and add the \lstinline|final| option.
\item
At the \textbf{publication stage}, you can remove the \lstinline|showeqnr| option, to display only referenced equation numbers.
\end{enumerate}

\section{References}
\label{sec:references}

Here is an equation without reference, but nevertheless numbered with the \lstinline|showeqnr| option on:
\begin{align}
f \colon A \to B
\end{align}

Here is a labeled equation, with label visible in the margin:
\begin{align}
\label{eq:important}
2 + 2 = 4
\end{align}

Note how the equation and section labels are printed in the margin.

\section{Theorems}

The following holds
\begin{theorem}
Sensible theorem, lemma and definition environments are already defined.
\end{theorem}

\section{Delimiters}

Some extra math delimiter commands are defined.
They accept an optional size argument (\lstinline|\big|, \lstinline|\Big|, etc.), or a star for automatic size adjustment.
For example, for \lstinline|\paren|, compare the following:
\begin{itemize}
\item \lstinline|\paren{\sqrt{x}+1}^2| \(\paren{\sqrt{x}+1}^2\)
\item \lstinline|\paren[\big]{\sqrt{x}+1}^2| \(\paren[\big]{\sqrt{x}+1}^2\)
\item \lstinline|\paren*{\sqrt{x}+1}^2| \(\paren*{\sqrt{x}+1}^2\)
\end{itemize}

The list of delimiters is:

\begin{itemize}
\item Parenthesis \lstinline|\paren|: \(\paren[\big]{\sqrt{x} + 1}^2\)
  \item Brackets \lstinline|\bracket|: \(\bracket[\big]{\frac{a}{2}+b}\)
\item Curly brace \lstinline|\cbrace| (but use \lstinline|\set| for sets): \(\cbrace{a+b}\)
\item Absolute value \lstinline|\abs|: \(\abs{x}\)
\item Norm \lstinline|\norm|: \(\norm{x}\)
\item Dual pairing, \lstinline|\pairing|: \(\pairing{\omega}{X}\)
  \item \lstinline|\spangle|: \(\spangle{E}\)
\item \lstinline|\scalprod|: \(\scalprod{\vec{u}}{\vec{v}}\)
\item Sets, \lstinline|\set|: \(S \coloneqq \set{1,2,\dotsc}\)
  \item Set comprehension \lstinline|\setc|: \(S' \coloneqq \setc{i \in S}{i \leq 10}\).
\end{itemize}

\section{Watermark}

There is also a nice watermark on the top of every page (compile several times).

\section{Todos}

\itodo[Me]{Here is an inline todo of some urgent issue to address}

\itodo*[Me]{Another, less urgent task, with the starred version of the \texttt{itodo} command.}

\section{Compatibility Options}

There are two extra options, \lstinline|noetex| and \lstinline|nomicrotype| which can be used along with classes which do not work well with those packages.

\section{Package List}

Here is a list of the major packages loaded by \texttt{allergy}.

Main packages:
\begin{itemize}
\item \texttt{hyperref}
\item \texttt{xcolor}
\item \texttt{amsmath}, \texttt{amssymb}, \texttt{amsthm}
\item \texttt{tikz}
\item \texttt{mathtools}
\item \texttt{etoolbox}
\item \texttt{xparse}
\end{itemize}

Draft mode packages:
\begin{itemize}
\item \texttt{todonotes}
\item \texttt{showkeys}
\end{itemize}

Some extra useful packages:
\begin{itemize}
\item \texttt{datetime}
\item \texttt{watermark} (optional)
\item \texttt{biblatex} (optional)
\end{itemize}

\end{document}
